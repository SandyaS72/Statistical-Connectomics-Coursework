\documentclass{article}

\usepackage{amsmath}
\usepackage{mathrsfs}

\title{Potential Model}
\author{akim1}
\date{February 25, 2015}


\begin{document}
\maketitle

The problem is formulated as follows: $\mathscr{G}_n=(V,X,Y,Z)$ where as defined
in class $V$ is the set of all nodes with edges, $X=(0,1)^{n\times n}$ and
$Y=(0,1)^n$ describing the excitatory or inhibitory nature of the neuron. $Z$
represents the orientation of the neuron discretized into four equally spaced
regions starting from 0 degree to 180 degrees.
$\mathit{SBM}_n^k=(\rho(z),\beta(z))$.

\section{Potential Model of $\mathscr{Z}$}
The simplest model of $\mathscr{Z}$ would a uniform distribution that assigns
equal probability to the four regions from 0 degree to 180 degrees. It seems
that a more educated distribution can be constructed, however infeasible, by
characterizing the actual orientation of all the neurons \textit{in vivo}.

\section{Potential Structure of $\rho(z)$ and $\beta(z)$}
As described in class, $\rho\in\Delta_k$ and $\beta\in(0,1)^{k\times k}$. The
simplest distribution for $\beta$ would be a Bernoulli distribution. The model
proposed above has no dependence on the orientation of the neurons. In the
respect to the graph being studied, this would not be a useful model as we
initially assumed that the connectivity of the graph has an inherent dependence
on the orientation of the neurons.
\end{document}
