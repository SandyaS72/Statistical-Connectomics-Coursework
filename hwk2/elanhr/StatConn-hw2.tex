%%%%%%%%%%%%%%%%%%%%%%%%%%%%%%%%%%%%%%%%%
% Short Sectioned Assignment
% LaTeX Template
% Version 1.0 (5/5/12)
%
% This template has been downloaded from:
% http://www.LaTeXTemplates.com
%
% Original author:
% Frits Wenneker (http://www.howtotex.com)
%
% License:
% CC BY-NC-SA 3.0 (http://creativecommons.org/licenses/by-nc-sa/3.0/)
%
%%%%%%%%%%%%%%%%%%%%%%%%%%%%%%%%%%%%%%%%%

%----------------------------------------------------------------------------------------
%	PACKAGES AND OTHER DOCUMENT CONFIGURATIONS
%----------------------------------------------------------------------------------------

\documentclass[paper=a4, fontsize=11pt]{scrartcl} % A4 paper and 11pt font size

\usepackage[T1]{fontenc} % Use 8-bit encoding that has 256 glyphs
\usepackage{fourier} % Use the Adobe Utopia font for the document - comment this line to return to the LaTeX default
\usepackage[english]{babel} % English language/hyphenation
\usepackage{amsmath,amsfonts,amsthm} % Math packages

\usepackage{lipsum} % Used for inserting dummy 'Lorem ipsum' text into the template
\usepackage{cite}

\usepackage{sectsty} % Allows customizing section commands
\allsectionsfont{\centering \normalfont\scshape} % Make all sections centered, the default font and small caps

\usepackage{fancyhdr} % Custom headers and footers
\pagestyle{fancyplain} % Makes all pages in the document conform to the custom headers and footers
\fancyhead{} % No page header - if you want one, create it in the same way as the footers below
\fancyfoot[L]{} % Empty left footer
\fancyfoot[C]{} % Empty center footer
\fancyfoot[R]{\thepage} % Page numbering for right footer
\renewcommand{\headrulewidth}{0pt} % Remove header underlines
\renewcommand{\footrulewidth}{0pt} % Remove footer underlines
\setlength{\headheight}{13.6pt} % Customize the height of the header

\numberwithin{equation}{section} % Number equations within sections (i.e. 1.1, 1.2, 2.1, 2.2 instead of 1, 2, 3, 4)
\numberwithin{figure}{section} % Number figures within sections (i.e. 1.1, 1.2, 2.1, 2.2 instead of 1, 2, 3, 4)
\numberwithin{table}{section} % Number tables within sections (i.e. 1.1, 1.2, 2.1, 2.2 instead of 1, 2, 3, 4)

\setlength\parindent{0pt} % Removes all indentation from paragraphs - comment this line for an assignment with lots of text

%----------------------------------------------------------------------------------------
%	TITLE SECTION
%----------------------------------------------------------------------------------------

\newcommand{\horrule}[1]{\rule{\linewidth}{#1}} % Create horizontal rule command with 1 argument of height

\title{	
\normalfont \normalsize 
%\horrule{0.5pt} \\[0.4cm] % Thin top horizontal rule
\large{Homework 2} \\ 
Elan Hourticolon-Retzler
% The assignment title
%\horrule{2pt} \\[0.5cm] % Thick bottom horizontal rule
}

%\author{Elan Hourticolon-Retzler} % Your name
%\textsc{} \\ [25pt] % Your university, school and/or department name(s)

\date{} % Today's date or a custom date

\begin{document}

\maketitle % Print the title

%----------------------------------------------------------------------------------------

\section*{Model Description for SBM}
{\bf Sample Space $\Xi$: (things we can observe)}

\begin{equation}
\begin{split}
G &\in \mathcal{G}\\
V &= \{ v_1,v_2,\dots, v_n \}\\
E &= \{0,1\}^{n\times n }\\
Y &= \{0,\dots, k\}\\
G &= (V, E,Y)
\end{split}
\end{equation}


{\bf Model:  $P = {p_\theta: \theta^{~} \in \Theta}$}

Our model parameters are the probability of being assigned to each group, and the probability of there being an edge between two vertices conditioned on their block membership.

\begin{equation}
\begin{split}
\rho &\in \triangle_k = \{ \theta_1,\theta_2,\dots,\theta_k \mid \sum \theta_i = 1 \}  \\
\beta &\in (0,1)^{k \times k}\\
\end{split}
\end{equation}


{\bf Action Space:  $A = {a, a_v, \dots}$  (what we are trying to do)}

In this model, our action space is assigning a vertex to a specific block.\\
\\
{\bf Loss Function $L: P \times A \rightarrow R^+$ }

The Loss function finds our cost of differing from another (possibly a gold standard) graph
A common rule is the Adjusted Rand Index or ARI:

\begin{equation}
\begin{split}
ARI = \frac{Index - ExceptedIndex}{MaxIndex - ExpectedIndex}
\end{split}
\end{equation}




{\bf Risk Function:	$R: L \times \Phi \times P \rightarrow R^+$}

A common Risk Function for SBM is simply expected loss ($E_p [L]$)





\end{document}
