\documentclass[12pt]{article}
\usepackage[margin=1in]{geometry}
\geometry{letterpaper}
\usepackage{graphicx}
\usepackage[hyphens]{url}
\usepackage{fancyhdr}
\pagestyle{fancy}
\usepackage{fixltx2e}
\usepackage{amsmath,amsfonts,amsthm,amssymb}
\usepackage{graphicx}
\usepackage{algorithm}
\usepackage{algorithmic}
\usepackage{url}
\usepackage[normalem]{ulem}
\usepackage[pdftex]{color}
\usepackage{varioref}
\usepackage{mathrsfs}
\usepackage{amsmath}
\labelformat{equation}{\textup{(#1)}}
\usepackage[sort&compress,colon,square,numbers]{natbib}
\usepackage{color}
\newcommand{\todo}[1]{{\color{red}{\it TODO: #1}}}
\newcommand{\jovo}[1]{{\color{green}{\it jovo: #1}}}
\newcommand{\will}[1]{{\color{blue}{\it will: #1}}}
\newcommand{\greg}[1]{{\color{cyan}{\it greg: #1}}}
\begin{document}
\bibliographystyle{plain}
\begin{center}\Large \bf EN.580.694: Statistical Connectomics \\ Final Project Proposal \end{center}
\begin{center} Michael Norris $\cdot$ \today \end{center}
\bigskip
\section*{Using a Random Dot Product Model to Approximate the (clustering?) of 
the C. Elegans Connectome}
\paragraph{Opportunity}
The Paper by \cite{2014PLoSO...997584P} contains an analysis of ERMM, SBM, [other two methods] for C. Elegans.  There's an opportunity to do the same level analysis comparing
ERMM as a baseline to a Random Dot Product Graph Model (RDP) on C. Elegans, and 
comparing the results.  If using this method provides statistically significant results, then it may be a useful tool for connectomics.
\paragraph{Challenge}
The challenge to this problem is formulating the problem correctly,
performing a correct statistical test on the connectome, and validating the
results.

\paragraph{Action}
Random Dot Product Graph Model is a graph model that can be used in place of 
other models.  I will test the Random Dot Product Graph Model using C. Elegans 
Data.
\paragraph{Resolution}
We will show whether the Random Dot Product Graph Model is able to cluster the 
connectome into similar modules as ERMM.
\paragraph{Future Work}
The results of this paper will give insight to when a Random Dot Product Graph 
Model could be used in a graph statistical problem.
\pagebreak
\subsection*{Statistical Decision Theoretic}
\begin{description}
\item[Sample Space]
The sample space is the connectome of C. Elegans.
\item[Model]
The model is the RDP Graph Model.  
ERMM will also be used in a separate test as a comparison.
\item[Action Space]
The Action Space is the distribution of all possible Random Dot Product Graphs.

\item[Decision Rule Class]
The decision rule is block assignments for each vertex.

\item[Loss Function]
The loss function is the Integrated Classification Likelihood so we can compare
across different blocks. \cite{2014PLoSO...997584P} equation (7).\\
$ICL(\mathscr{M}_{q}) = max\log | \mathscr{L}(x, \tilde{z}|\mathscr{M}_{q};\phi) -
\frac{1}{2}\frac{Q(Q+1)}{2} \log\left[\frac{n(n-1)}{2}\right] - \frac{Q-1}{2}
\log\left[n\right]$\\
\item[Risk Function]
The risk function is the expected value of the loss function 
$R = E[\mathscr{l}]$.

\end{description}
\bibliography{master}
\end{document} 
